%%%%%%%%%%%%%%%%%%%%%%%%%%%%%%%%%%%%%%%%%
% Beamer Presentation
% LaTeX Template
% Version 1.0 (10/11/12)
%
% This template has been downloaded from:
% http://www.LaTeXTemplates.com
%
% License:
% CC BY-NC-SA 3.0 (http://creativecommons.org/licenses/by-nc-sa/3.0/)
%
%%%%%%%%%%%%%%%%%%%%%%%%%%%%%%%%%%%%%%%%%

\documentclass{beamer}

%-------------------------------------------------------------------
%	PACKAGES AND THEMES
%-------------------------------------------------------------------

%%% PACKAGES
\usepackage[utf8]{inputenc}  % italian symbols.
\usepackage[T1]{fontenc}     % define T1 charset for out files.
\usepackage[italian]{babel}  % italian latex typo conventions.
\usepackage{csquotes}        % needed by babel.
\usepackage{amsmath}         % math features.
\usepackage{amsthm}          % math theorems.
\usepackage{amssymb}         % math symbols.
\usepackage{graphicx}        % images managing.
\usepackage{booktabs}        % Allows the use of \toprule, \midrule and \bottomrule in tables
\usepackage{algorithm}       % algorithm block.
\usepackage{algcompatible}
\usepackage{algpseudocode}   % style for (autoimported) package algorithmicx.

%%% MODE
\mode<presentation> {
% The Beamer class comes with a number of default slide themes
% which change the colors and layouts of slides. Below this is a
% list of all the themes, uncomment each in turn to see what they
% look like.

%\usetheme{default}
%\usetheme{AnnArbor}
%\usetheme{Antibes}
%\usetheme{Bergen}        % commented left
%\usetheme{Berkeley}      % menu left
%\usetheme{Berlin}
%\usetheme{Boadilla}      % nice, no top menu
%\usetheme{CambridgeUS}   % nice menu and footer
%\usetheme{Copenhagen}
%\usetheme{Darmstadt}
%\usetheme{Dresden}
%\usetheme{Frankfurt}     % top bullets, no info below
%\usetheme{Goettingen}    % no
%\usetheme{Hannover}      % no
%\usetheme{Ilmenau}       % not bad: bullets, sections, ... but too many rows
%\usetheme{JuanLesPins}   % tree menu
%\usetheme{Luebeck}       % very nice menu
\usetheme{Madrid}        % default
%\usetheme{Malmoe}
%\usetheme{Marburg}
%\usetheme{Montpellier}
%\usetheme{PaloAlto}
%\usetheme{Pittsburgh}    % clean
%\usetheme{Rochester}
%\usetheme{Singapore}     % not bad, no info below
%\usetheme{Szeged}
%\usetheme{Warsaw}

% As well as themes, the Beamer class has a number of color themes
% for any slide theme. Uncomment each of these in turn to see how it
% changes the colors of your current slide theme.

%\usecolortheme{albatross}     % horrible
%\usecolortheme{beaver}        % red blue
%\usecolortheme{beetle}        % horrible
\usecolortheme{crane}         % nice: yellow, orange
%\usecolortheme{dolphin}       % simil default
%\usecolortheme{dove}          % black and white
%\usecolortheme{fly}           % horrible
%\usecolortheme{lily}          % nice, clean blue
%\usecolortheme{orchid}        % like default
%\usecolortheme{rose}          % like default, very good
%\usecolortheme{seagull}       % grey
%\usecolortheme{seahorse}      % light lavanda
%\usecolortheme{whale}         % like default
%\usecolortheme{wolverine}     % yello blue

% To remove the footer line in all slides uncomment this line
%\setbeamertemplate{footline}

% To replace the footer line in all slides with a simple slide 
% count uncomment this line
%\setbeamertemplate{footline}[page number]

% To remove the navigation symbols from the bottom of all 
% slides uncomment this line
\setbeamertemplate{navigation symbols}{} 

% blocks
%\setbeamertemplate{blocks}[rounded][shadow=true]

} % /mode<presentation>

% top menu
\useoutertheme[subsection=false]{miniframes}

%\setbeamercolor{block title}{use=structure,fg=white,bg=blue!75!black}
%\setbeamercolor{block body}{use=structure,fg=black,bg=white!20!white}

%%% CONFIGURATIONS
%% Special letters denoting sets and algebras.
\providecommand*{\Nset}{\mathbb{N}}             % Naturals
\providecommand*{\Qset}{\mathbb{Q}}             % Rationals
\providecommand*{\Zset}{\mathbb{Z}}             % Integers
\providecommand*{\Rset}{\mathbb{R}}             % Reals

%% Calligraphic alphabet.
\newcommand*{\calA}{\ensuremath{\mathcal{A}}}
\newcommand*{\calB}{\ensuremath{\mathcal{B}}}
\newcommand*{\calC}{\ensuremath{\mathcal{C}}}
\newcommand*{\calD}{\ensuremath{\mathcal{D}}}
\newcommand*{\calE}{\ensuremath{\mathcal{E}}}
\newcommand*{\calF}{\ensuremath{\mathcal{F}}}
\newcommand*{\calG}{\ensuremath{\mathcal{G}}}
\newcommand*{\calH}{\ensuremath{\mathcal{H}}}
\newcommand*{\calI}{\ensuremath{\mathcal{I}}}
\newcommand*{\calJ}{\ensuremath{\mathcal{J}}}
\newcommand*{\calK}{\ensuremath{\mathcal{K}}}
\newcommand*{\calL}{\ensuremath{\mathcal{L}}}
\newcommand*{\calM}{\ensuremath{\mathcal{M}}}
\newcommand*{\calN}{\ensuremath{\mathcal{N}}}
\newcommand*{\calO}{\ensuremath{\mathcal{O}}}
\newcommand*{\calP}{\ensuremath{\mathcal{P}}}
\newcommand*{\calQ}{\ensuremath{\mathcal{Q}}}
\newcommand*{\calR}{\ensuremath{\mathcal{R}}}
\newcommand*{\calS}{\ensuremath{\mathcal{S}}}
\newcommand*{\calT}{\ensuremath{\mathcal{T}}}
\newcommand*{\calU}{\ensuremath{\mathcal{U}}}
\newcommand*{\calV}{\ensuremath{\mathcal{V}}}
\newcommand*{\calW}{\ensuremath{\mathcal{W}}}
\newcommand*{\calX}{\ensuremath{\mathcal{X}}}
\newcommand*{\calY}{\ensuremath{\mathcal{Y}}}
\newcommand*{\calZ}{\ensuremath{\mathcal{Z}}}

% **********************************************************
% Luca Parolari <luca.parolari23@gmail.com>
% -- macros for bachelor thesis <TITLE> <url_here>

\newcommand*{\clpset}{CLP(\calS\calE\calT)}                % CLP(SET)
\newcommand*{\satset}{SAT\textsubscript{\calS\calE\calT}}  % SAT_SET
\newcommand*{\calset}{\calS\calE\calT}                     % SET
\newcommand*{\satbr}{SAT\textsubscript{\calB\calR}}        % SAT_BR
\newcommand*{\lbr}{\textit{L}\textsubscript{BR}}           % L_BR
\newcommand*{\lset}{\textit{L}\textsubscript{\calS\calE\calT}} % L_SET
\newcommand*{\setlog}{$\{$log$\}$}                         % {log}
\newcommand*{\jsetl}{JSetL}                                 % JSetL

% Dotted letters
\newcommand*{\dotA}{\ensuremath{\dot{A}}}
\newcommand*{\dotB}{\ensuremath{\dot{B}}}
\newcommand*{\dotC}{\ensuremath{\dot{C}}}
\newcommand*{\dotx}{\ensuremath{\dot{x}}}

\newcommand*{\fixme}[1]{\footnote{\textbf{FIXME:} {#1}}}
\newcommand*{\todo}[1]{\footnote{\textbf{TODO:} {#1}}}




%-------------------------------------------------------------------
%	TITLE PAGE
%-------------------------------------------------------------------

% The short title appears at the bottom of every slide, 
% the full title is only on the title page
\title[
	\lset{} in Picat
]{
	Progettazione e implementazione
	in Picat di un risolutore per vincoli insiemistici}

\author[Luca \textsc{Parolari}]{
\textit{Candidato:} Luca \textsc{Parolari} \\
\textit{Relatore:} Gianfranco \textsc{Rossi}
}
\institute[UNIPR]
{
	Università di Parma \\
	Dipartimento di Scienze Matematiche Fisiche e Informatiche \\
	Corso di Laurea in Informatica
}
\date{26 settembre 2019}


\begin{document}

\begin{frame}
\titlepage % Print the title page as the first slide
\end{frame}

\begin{frame}
\frametitle{Panoramica della presentazione}
%%% manual toc
%\begin{itemize}
%	\item Il linguaggio \lset{}
%\end{itemize}

%%% automatic toc
\tableofcontents
\end{frame}

%-------------------------------------------------------------------
%	PRESENTATION SLIDES
%-------------------------------------------------------------------

%------------------------------------------------
\section{\lset{}}
%------------------------------------------------

\begin{frame}
  \frametitle{Il linguaggio \lset{}}
  \begin{block}{Definizione (\lset{})}
    \lset{} è un linguaggio basato su vincoli per esprimere e
    risolvere formule del primo ordine sull'universo degli insiemi
    finiti.
  \end{block}
\end{frame}

%------------------------------------------------

\begin{frame}
  \frametitle{Sintassi}
  \begin{block}{Sintassi}
    Insiemi di simboli costituenti di \lset{}:
    \begin{itemize}
    \item $\calF$, insieme di costanti e simboli di funzione;
    \item $\prod_C = \{ =, \in, un, disj, set \}$;
    \item $\calV$, insieme numerabile di varabili.
    \end{itemize}
  \end{block}
  \medskip
  \textbf{Esempi}
  \begin{itemize}
  \item $1$, $a$, $f(\ldots)$, etc.
  \item $un(\cdot, \cdot, \cdot)$, etc.
  \item $A$, $B$, $X$, $Y$, $Var$, etc.
  \end{itemize}
\end{frame}

%------------------------------------------------

\begin{frame}
  \frametitle{Semantica informale}
  \begin{block}{Semantica informale}
    Semantica intuitiva dei simboli in $\Sigma_{\lset}$:
    \begin{itemize}
    \item $\emptyset$ rappresenta l’insieme vuoto;
    \item $\{\cdot\mid\cdot\}$ rappresenta il costruttore di insiemi
      definito come $\{t\mid s\} = \{t\}\ \cup s$;
    \item il predicato $=$ rappresenta la relazione di uguaglianza;
    \item il predicato \textit{in} rappresenta la relazione di
      appartenenza;
    \item il predicato \textit{un} rappresenta la relazione di unione;
      insiemistica definita come: $un(r,s,t) = true
      \Longleftrightarrow t = r \cup s$;
    \item il predicato \textit{disj} rappresenta la relazione di
      disgiunzione insiemistica definita come $disj(r,s) = true
      \Longleftrightarrow r \cap s = \emptyset$;
    \item il predicato \textit{set} controlla che il termine sia un
      insieme.
    \end{itemize}
  \end{block}
\end{frame}

%------------------------------------------------

\begin{frame}
  \frametitle{Esempi} Il linguaggio \lset{} consente di combinare i
  simboli in \lset{}-\textit{termini}, \lset{}-\textit{vincoli} ed
  \lset{}-\textit{formule}.
  \begin{columns}[c]
    \column{.45\textwidth}
    \begin{exampleblock}{Esempio (\lset{}-\textit{termini})}
      \begin{itemize}
      \item $\{1 \mid \emptyset \} \equiv \{1\}$
      \item $\{1 \mid \{2 \mid \emptyset \} \} \equiv \{1,2\}$
      \item $\{1 \mid \{2 \mid X \} \} \equiv \{1,2 \mid X \}$
      \end{itemize}
    \end{exampleblock}
    \begin{exampleblock}{Esempio (\lset{}-\textit{vincoli})}
      \begin{itemize}
      \item $X = 1$
      \item $un(\{1\}, \emptyset, R)$
      \end{itemize}
    \end{exampleblock}
    \column{.45\textwidth}
    \begin{exampleblock}{Esempio (\lset{}-\textit{formule})}
      \begin{itemize}
      \item $1\ in\ R \land 1\ nin\ S \land un(R,S,T) \land T = \{X\}$
      \item $(X\ in\ S \land X\ neq\ 1) \lor (X\ nin\ S \land X = 1)$
      \end{itemize}
    \end{exampleblock}
  \end{columns}
\end{frame}

%------------------------------------------------

\begin{frame}
  \frametitle{Il solver \satset{}}
  % The "c" option specifies centered vertical alignment while the "t"
  % option is used for top vertical alignment
  \begin{columns}[c]

    \column{.5\textwidth} % Left column and width
    %\begin{algorithm}
    %	\caption{Procedura \satset{}}
    \begin{algorithmic}[1]
      \Procedure{\satset{$(C)$}}{}
      \State $C \gets $ \texttt{sort\_infer($C$)};
      \Repeat
      \State $C'$ $\gets C;$
      \Repeat
      \State $C'' \gets C;$
      \State $C \gets $ \texttt{STEP($C$)};
      \Until {$C$ = $C''$;}
      \State $C \gets $ \texttt{remove\_neq($C$)};
      \Until {$C'$ = $C$;}
      \State
      \State\Return $C$;
      \EndProcedure
    \end{algorithmic}
    %	\label{alg:pseudo_satset}
    %\end{algorithm}

    \column{.45\textwidth} % Right column and width
    La procedura \satset{} è la procedura di risoluzione dei vincoli,
    che in \texttt{sort\textunderscore infer} aggiunge i vincoli sui
    tipi, in \texttt{remove\textunderscore neq} rimuove un certo tipo
    di $\neq$-\textit{constraints}, mentre in \texttt{STEP} applica le
    \textbf{regole di riscrittura}.
  \end{columns}
\end{frame}

%------------------------------------------------
\section{Second Section}
%------------------------------------------------

\begin{frame}
\frametitle{Table}
\begin{table}
\begin{tabular}{l l l}
\toprule
\textbf{Treatments} & \textbf{Response 1} & \textbf{Response 2}\\
\midrule
Treatment 1 & 0.0003262 & 0.562 \\
Treatment 2 & 0.0015681 & 0.910 \\
Treatment 3 & 0.0009271 & 0.296 \\
\bottomrule
\end{tabular}
\caption{Table caption}
\end{table}
\end{frame}

%------------------------------------------------

\begin{frame}
\frametitle{Theorem}
\begin{theorem}[Mass--energy equivalence]
$E = mc^2$
\end{theorem}
\end{frame}

%------------------------------------------------

\begin{frame}[fragile] % Need to use the fragile option when verbatim is used in the slide
\frametitle{Verbatim}
\begin{example}[Theorem Slide Code]
\begin{verbatim}
\begin{frame}
\frametitle{Theorem}
\begin{theorem}[Mass--energy equivalence]
$E = mc^2$
\end{theorem}
\end{frame}\end{verbatim}
\end{example}
\end{frame}

%------------------------------------------------

\begin{frame}
\frametitle{Figure}
Uncomment the code on this slide to include your own image from the same directory as the template .TeX file.
%\begin{figure}
%\includegraphics[width=0.8\linewidth]{test}
%\end{figure}
\end{frame}

%------------------------------------------------

\begin{frame}[fragile] % Need to use the fragile option when verbatim is used in the slide
\frametitle{Citation}
An example of the \verb|\cite| command to cite within the presentation:\\~

This statement requires citation \cite{p1}.
\end{frame}

%------------------------------------------------

\begin{frame}
\frametitle{References}
\footnotesize{
\begin{thebibliography}{99} % Beamer does not support BibTeX so references must be inserted manually as below
\bibitem[Smith, 2012]{p1} John Smith (2012)
\newblock Title of the publication
\newblock \emph{Journal Name} 12(3), 45 -- 678.
\end{thebibliography}
}
\end{frame}

%------------------------------------------------

\begin{frame}
  \Huge{\centerline{Grazie per l'attenzione}}
\end{frame}

%-------------------------------------------------------------------

\end{document}
