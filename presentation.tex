%%%%%%%%%%%%%%%%%%%%%%%%%%%%%%%%%%%%%%%%%
% Beamer Presentation
% LaTeX Template
% Version 1.0 (10/11/12)
%
% This template has been downloaded from:
% http://www.LaTeXTemplates.com
%
% License:
% CC BY-NC-SA 3.0 (http://creativecommons.org/licenses/by-nc-sa/3.0/)
%
%%%%%%%%%%%%%%%%%%%%%%%%%%%%%%%%%%%%%%%%%

\documentclass{beamer}

%-------------------------------------------------------------------
%	PACKAGES AND THEMES
%-------------------------------------------------------------------

%%% PACKAGES
\usepackage[utf8]{inputenc}  % italian symbols.
\usepackage[T1]{fontenc}     % define T1 charset for out files.
\usepackage[italian]{babel}  % italian latex typo conventions.
\usepackage{csquotes}        % needed by babel.
\usepackage{amsmath}         % math features.
\usepackage{amsthm}          % math theorems.
\usepackage{amssymb}         % math symbols.
\usepackage{graphicx}        % images managing.
\usepackage{booktabs}        % Allows the use of \toprule, \midrule and \bottomrule in tables
\usepackage{algorithm}       % algorithm block.
\usepackage{algcompatible}
\usepackage{algpseudocode}   % style for (autoimported) package algorithmicx.

%%% MODE
\mode<presentation> {
% The Beamer class comes with a number of default slide themes
% which change the colors and layouts of slides. Below this is a
% list of all the themes, uncomment each in turn to see what they
% look like.

%\usetheme{default}
%\usetheme{AnnArbor}
%\usetheme{Antibes}
%\usetheme{Bergen}        % commented left
%\usetheme{Berkeley}      % menu left
%\usetheme{Berlin}
%\usetheme{Boadilla}      % nice, no top menu
%\usetheme{CambridgeUS}   % nice menu and footer
%\usetheme{Copenhagen}
%\usetheme{Darmstadt}
%\usetheme{Dresden}
%\usetheme{Frankfurt}     % top bullets, no info below
%\usetheme{Goettingen}    % no
%\usetheme{Hannover}      % no
%\usetheme{Ilmenau}       % not bad: bullets, sections, ... but too many rows
%\usetheme{JuanLesPins}   % tree menu
%\usetheme{Luebeck}       % very nice menu
\usetheme{Madrid}        % default
%\usetheme{Malmoe}
%\usetheme{Marburg}
%\usetheme{Montpellier}
%\usetheme{PaloAlto}
%\usetheme{Pittsburgh}    % clean
%\usetheme{Rochester}
%\usetheme{Singapore}     % not bad, no info below
%\usetheme{Szeged}
%\usetheme{Warsaw}

% As well as themes, the Beamer class has a number of color themes
% for any slide theme. Uncomment each of these in turn to see how it
% changes the colors of your current slide theme.

%\usecolortheme{albatross}     % horrible
%\usecolortheme{beaver}        % red blue
%\usecolortheme{beetle}        % horrible
\usecolortheme{crane}         % nice: yellow, orange
%\usecolortheme{dolphin}       % simil default
%\usecolortheme{dove}          % black and white
%\usecolortheme{fly}           % horrible
%\usecolortheme{lily}          % nice, clean blue
%\usecolortheme{orchid}        % like default
%\usecolortheme{rose}          % like default, very good
%\usecolortheme{seagull}       % grey
%\usecolortheme{seahorse}      % light lavanda
%\usecolortheme{whale}         % like default
%\usecolortheme{wolverine}     % yello blue

\usefonttheme{professionalfonts}

% To remove the footer line in all slides uncomment this line
%\setbeamertemplate{footline}

% To replace the footer line in all slides with a simple slide 
% count uncomment this line
%\setbeamertemplate{footline}[page number]

% To remove the navigation symbols from the bottom of all 
% slides uncomment this line
\setbeamertemplate{navigation symbols}{} 

% blocks
%\setbeamertemplate{blocks}[rounded][shadow=true]

} % /mode<presentation>

% top menu
\useoutertheme[subsection=false]{miniframes}

%\setbeamercolor{block title}{use=structure,fg=white,bg=blue!75!black}
%\setbeamercolor{block body}{use=structure,fg=black,bg=white!20!white}

% step by step
\setbeamercovered{transparent}

%%% CONFIGURATIONS
%% Special letters denoting sets and algebras.
\providecommand*{\Nset}{\mathbb{N}}             % Naturals
\providecommand*{\Qset}{\mathbb{Q}}             % Rationals
\providecommand*{\Zset}{\mathbb{Z}}             % Integers
\providecommand*{\Rset}{\mathbb{R}}             % Reals

%% Calligraphic alphabet.
\newcommand*{\calA}{\ensuremath{\mathcal{A}}}
\newcommand*{\calB}{\ensuremath{\mathcal{B}}}
\newcommand*{\calC}{\ensuremath{\mathcal{C}}}
\newcommand*{\calD}{\ensuremath{\mathcal{D}}}
\newcommand*{\calE}{\ensuremath{\mathcal{E}}}
\newcommand*{\calF}{\ensuremath{\mathcal{F}}}
\newcommand*{\calG}{\ensuremath{\mathcal{G}}}
\newcommand*{\calH}{\ensuremath{\mathcal{H}}}
\newcommand*{\calI}{\ensuremath{\mathcal{I}}}
\newcommand*{\calJ}{\ensuremath{\mathcal{J}}}
\newcommand*{\calK}{\ensuremath{\mathcal{K}}}
\newcommand*{\calL}{\ensuremath{\mathcal{L}}}
\newcommand*{\calM}{\ensuremath{\mathcal{M}}}
\newcommand*{\calN}{\ensuremath{\mathcal{N}}}
\newcommand*{\calO}{\ensuremath{\mathcal{O}}}
\newcommand*{\calP}{\ensuremath{\mathcal{P}}}
\newcommand*{\calQ}{\ensuremath{\mathcal{Q}}}
\newcommand*{\calR}{\ensuremath{\mathcal{R}}}
\newcommand*{\calS}{\ensuremath{\mathcal{S}}}
\newcommand*{\calT}{\ensuremath{\mathcal{T}}}
\newcommand*{\calU}{\ensuremath{\mathcal{U}}}
\newcommand*{\calV}{\ensuremath{\mathcal{V}}}
\newcommand*{\calW}{\ensuremath{\mathcal{W}}}
\newcommand*{\calX}{\ensuremath{\mathcal{X}}}
\newcommand*{\calY}{\ensuremath{\mathcal{Y}}}
\newcommand*{\calZ}{\ensuremath{\mathcal{Z}}}

% **********************************************************
% Luca Parolari <luca.parolari23@gmail.com>
% -- macros for bachelor thesis <TITLE> <url_here>

\newcommand*{\clpset}{CLP(\calS\calE\calT)}                % CLP(SET)
\newcommand*{\satset}{SAT\textsubscript{\calS\calE\calT}}  % SAT_SET
\newcommand*{\calset}{\calS\calE\calT}                     % SET
\newcommand*{\satbr}{SAT\textsubscript{\calB\calR}}        % SAT_BR
\newcommand*{\lbr}{\textit{L}\textsubscript{BR}}           % L_BR
\newcommand*{\lset}{\textit{L}\textsubscript{\calS\calE\calT}} % L_SET
\newcommand*{\setlog}{$\{$log$\}$}                         % {log}
\newcommand*{\jsetl}{JSetL}                                 % JSetL

% Dotted letters
\newcommand*{\dotA}{\ensuremath{\dot{A}}}
\newcommand*{\dotB}{\ensuremath{\dot{B}}}
\newcommand*{\dotC}{\ensuremath{\dot{C}}}
\newcommand*{\dotx}{\ensuremath{\dot{x}}}

\newcommand*{\fixme}[1]{\footnote{\textbf{FIXME:} {#1}}}
\newcommand*{\todo}[1]{\footnote{\textbf{TODO:} {#1}}}




%-------------------------------------------------------------------
%	TITLE PAGE
%-------------------------------------------------------------------

% The short title appears at the bottom of every slide, 
% the full title is only on the title page
\title[
  \lset{} in Picat
]{
  Progettazione e implementazione
  in Picat di un risolutore per vincoli insiemistici}

\author[Luca \textsc{Parolari}]{
  \textit{Candidato:} Luca Parolari \\
  \textit{Relatore:} Gianfranco Rossi
}

\institute[UNIPR]
{
  Università di Parma \\
  Dipartimento di Scienze Matematiche, Fisiche e Informatiche \\
  Corso di Laurea in Informatica
}
\date{26 settembre 2019}



\begin{document}

%------------------------------------------------

\begin{frame}
  \titlepage
\end{frame}

%------------------------------------------------

\begin{frame}
  \frametitle{Panoramica della presentazione}
  \tableofcontents
\end{frame}

%-------------------------------------------------------------------
%	PRESENTATION SLIDES
%-------------------------------------------------------------------

%------------------------------------------------
\section{\lset{}}
%------------------------------------------------

\begin{frame}
  \frametitle{Il linguaggio \lset{}}
  \begin{block}{Definizione (\lset{})}
    \lset{} è un linguaggio basato su vincoli per esprimere e
    risolvere formule del primo ordine sull'universo degli insiemi
    finiti.
  \end{block}
  \begin{itemize}
  	\item Insiemi molto generali: \emph{untyped hereditarily finite hybrid sets}.
  	\item Manipolazione tramite usuali operatori su insiemi.
  \end{itemize}
\end{frame}

%------------------------------------------------

\begin{frame}
  \frametitle{Sintassi}
  \begin{block}{Sintassi}
    Insiemi di simboli costituenti di \lset{}:
    \begin{itemize}
    \item $\calF$, insieme di costanti e simboli di funzione, tra cui $\emptyset$ e $\{ \cdot, \cdot \}$;
    \item $\prod_C = \{ =, in, un, disj, set \}$;
    \item $\calV$, insieme numerabile di variabili.
    \end{itemize}
  \end{block}
  \medskip
  \textbf{Esempi}
  \begin{itemize}
  \item $1$, $a$, $f(\ldots)$, $\{1|\{2|\emptyset\}\} \equiv \{1,2\}$ etc.
  \item $X = 5$, $un(\{7\}, \emptyset, S)$\footnote{La semantica di $un(A,B,S)$ è $S = A \cup B $}, etc.
  \item $(X\ in\ S \land X\ neq\ 1) \lor (X\ nin\ S \land X = 1)$, etc.
  \end{itemize}
\end{frame}

%------------------------------------------------

\begin{frame}
  \frametitle{Il solver \satset{}}
  \begin{columns}[c]

    \column{.45\textwidth} % Left column and width
    \begin{itemize}
    \item La procedura \satset{} è la procedura di decisione per i
      vincoli.
    \item \satset{} utilizza riscritture sintattiche per risolvere le
      formule.
    \end{itemize}

    \column{.5\textwidth} % Right column and width
    \begin{exampleblock}{Esempio (Formula insoddisfacibile)}
      $X = \{\} \land X\ neq\ \{\}$
      
      Nessuna soluzione.
    \end{exampleblock}
    \begin{exampleblock}{Esempio (Formula soddisfacibile)}
      $1\ nin\ \{A, B\}$
      
      Una soluzione: $A \neq 1 \land B \neq 1$
    \end{exampleblock}
  \end{columns}
\end{frame}

%------------------------------------------------
\section{Picat}
%------------------------------------------------

\begin{frame}
  \frametitle{Il linguaggio Picat}
  \begin{block}{Picat}
    Picat è un linguaggio di programmazione multiparadigma basato
    sulla logica dei predicati del primo ordine per applicazioni
    \emph{general-purpose}.
  \end{block}

  \medskip

  \textbf{Autori}: Neng-Fa Zhou e Jonathan Fruhman.
  
  \textbf{Stato dell'arte}:
  \begin{itemize}
  \item Inizio dello sviluppo a \emph{maggio 2013}.
  \item Rilascio della prima versione stabile 1.0 ad \emph{aprile
    2015}.
  \item Rilascio della versione 2.0 nel \emph{novembre 2017}.
  \item Ultima versione 2.6\#2 rilasciata a \emph{marzo 2019}.
  \end{itemize}
\end{frame}

\begin{frame}
  \frametitle{Il linguaggio Picat} \framesubtitle{Caratteristiche
    principali}
  \begin{itemize}
  \item \textbf{P}attern-matching, predicati e funzioni sono definiti
    tramite pattern-matching.
  \item \textbf{I}ntuitive, sono forniti comandi utilizzati nei comuni
    linguaggi imperativi.
  \item \textbf{C}onstraints, la programmazione con vincoli è
    supportata nativamente.
  \item \textbf{A}ctors, viene fornito il supporto per attori
    event-driven.
  \item \textbf{T}abling, viene fornito un sistema di caching per la
    memorizzazione di risultati parziali.
  \end{itemize}
\end{frame}

%------------------------------------------------

\begin{frame}[fragile]
  \frametitle{Esempi (1/2)}
  \framesubtitle{Approccio logic programming}
  \begin{exampleblock}{Esempio (Successione di Fibonacci)}
\begin{verbatim}
fib(0,F) => F=0.
fib(1,F) => F=1.
fib(N,F),N>1 => fib(N-1,F1), fib(N-2,F2), F=F1+F2.
fib(N,F) => throw $error(wrong_argument,fib,N).
\end{verbatim}
  \end{exampleblock}
\end{frame}

%------------------------------------------------

\begin{frame}[fragile]
  \frametitle{Esempi (2/2)}
  \framesubtitle{Approccio imperative programming}
  \begin{exampleblock}{Esempio (Successione di Fibonacci)}
\begin{verbatim}
fib(N) = F =>
    if (N = 0) then
        F = 0
    elseif (N = 1) then
        F = 1
    elseif N > 1 then
        F = fib(N-1) + fib(N-2)
    else 
        throw $error(wrong_argument,fib,N)
    end.
\end{verbatim}
  \end{exampleblock}
\end{frame}

%------------------------------------------------

\begin{frame}[fragile]
  \frametitle{Picat vs. Prolog}
  \begin{table}
    \begin{tabular}{l l l}
      & \textbf{Prolog} & \textbf{Picat} \\
      \addlinespace
      \textbf{Predicati dinamici} & Sì & No \\
      \textbf{Operatori definiti da utente} & Sì & No \\
      \textbf{Selezione regole applicabili} & Unificazione & Pattern-matching \\
      \textbf{Backtracking} & Sì (default) & Sì \\
      \textbf{Comandi/costrutti di controllo} & No & Si
    \end{tabular}
  \end{table}
\end{frame}

%------------------------------------------------

\begin{frame}[fragile]
  \frametitle{Unificazione e pattern-matching}
  
  Il predicato \texttt{p(X, X)} ha comportamenti diversi in Prolog e
  Picat.
  \begin{columns}[c]
    
    \column{0.45\textwidth}
    \begin{exampleblock}{Esempio (Unificazione)}
\begin{verbatim}
% Prolog
p(X, X) :- true.

?- p(1, 1).
true
?- p(1, Z).
Z = 1
\end{verbatim}
    \end{exampleblock}
    
    \column{0.45\textwidth}
    \begin{exampleblock}{Esempio (Pattern-matching)}
\begin{verbatim}
% Picat
p(X, X) => true.

> p(1, 1).
true
> p(1, Z).
false
\end{verbatim}
    \end{exampleblock}
  \end{columns}

  \medskip

  In Picat per avere lo stesso comportamento del Prolog bisogna
  scrivere \texttt{p(X, Y) => X = Y}, forzando l'unificazione.
\end{frame}

%------------------------------------------------

\begin{frame}[fragile]
  \frametitle{Backtracking}
  
  \begin{exampleblock}{Esempio (Gestione backtracking)}
\begin{verbatim}
% Prolog 
member(X,[X|_]).
member(X,[_|T]) :- member(X,T).

% Picat
member(X,[Y|_]) ?=> X = Y.
member(X,[_|T]) => member(X,T).
\end{verbatim}
  \end{exampleblock}

  \begin{itemize}
  \item In Picat è necessario indicare esplicitamente una regola
    \emph{backtrackable}.
  \item Una regola \emph{non-backtrackable} in Picat è come una regola
    Prolog seguita da \emph{cut} (!).
  \end{itemize}
  
\end{frame}

%------------------------------------------------

\begin{frame}[fragile]
  \frametitle{Costrutti} 
  Picat fornisce numerosi costrutti di controllo (\texttt{foreach}, \texttt{while}, \texttt{if-then-else}, etc.), funzioni e predicati \emph{built-in} e l'assegnamento (\texttt{:=}).
  \begin{exampleblock}{Esempio (Costrutti)}
\begin{verbatim}
show(List) =>
    I = 1,
    foreach(X in List)
        printf("%s: %s%n", 
               I.to_string(), 
               X.to_string()),
        I := I + 1
    end.
\end{verbatim}
  \end{exampleblock}
\end{frame}


%------------------------------------------------
\section{\lset{} in Picat}
%------------------------------------------------

\begin{frame}
  \frametitle{\lset{} in Picat}
  \framesubtitle{Linguaggio concreto di input per il solver}
  \begin{itemize}
  \item Adozione di una sintassi semplificata per il linguaggio di input.
  \end{itemize}
  \begin{table}
    \begin{tabular}{l l l}
      & \textbf{\lset{}} & \textbf{Impl. Picat}\\
      \addlinespace
      \textbf{Insiemi} & $\{T_1, \ldots, T_n \mid X\}$ & $[T_1, \ldots, T_n \mid X]$ \\
      \textbf{Vincoli} & $T_1 = T_2$ & $eq(T_1, T_2)$ \\
      \textbf{Formule} & $T_1 \land \ldots \land T_n$ & $[T_1, \ldots, T_n]$ \\
    \end{tabular}
  \end{table}
\end{frame}

%------------------------------------------------

\begin{frame}
  \frametitle{Architettura del sistema}
  \begin{itemize}
  \item Architettura a moduli:
    \begin{itemize}
    \item solver.pi
    \item commands.pi
    \item lset.pi
    \item global.pi
    \item log\textunderscore h.pi
    \item prompt.pi
    \item assert.pi
    \end{itemize}
  \item Approccio OOP.
  \end{itemize}
\end{frame}

%------------------------------------------------

\begin{frame}
  \frametitle{Il solver}
  \begin{itemize}
  \item Implementa la procedura decisionale \satset{}.
  \item Se le soluzioni esistono, deve essere in grado di mostrarle tutte.
  \item Riscrive i vincoli fintanto che non è raggiunto il punto fisso.
  \item La formula finale ottenuta è \textbf{irriducibile}, \textbf{semplificata} ed \textbf{equivalente} alla formula originale.
  \end{itemize}
\end{frame}

%------------------------------------------------

\begin{frame}[fragile]
  \frametitle{Il solver}
  \framesubtitle{Il predicato solve/2}
  
  \begin{columns}[c]
    \column{0.4\textwidth}
    \begin{itemize}
    \item Interfaccia del solver.
    \item Implementa il ciclo che riscrive la formula fino al punto fisso.
    \item Il predicato \texttt{sat/2} seleziona ed esegue le regole di riscrittura.
    \end{itemize}

    \column{0.5\textwidth}  
    \begin{exampleblock}{Esempio (Predicato solve/2)}
\begin{verbatim}
solve(C, CC) =>
    CN = [],
    foreach (T in C)
        sat(T, C1),
        CN := CN ++ C1,
    end,
    CN := CN.flatten(),
    
    if C == CN
    then CC = CN
    else solve(CN, CC)
    end.
\end{verbatim}
    \end{exampleblock}	
  \end{columns}
\end{frame}

%------------------------------------------------
\section{Implementazione}
%------------------------------------------------

%------------------------------------------------

\begin{frame}[fragile]
  \frametitle{Implementazione regole di riscrittura} L'implementazione
  in Picat è semplice e molto vicina alla descrizione matematica.
  \begin{columns}[c]
    \column{0.45\textwidth}
    \begin{exampleblock}{Esempio (Desc. matematica)}
      Regola $=_2$:
      \[
      \text{Se }t \not\in \calV, t = \dotx \to \dotx = t
      \]
    \end{exampleblock}
    
    \column{0.45\textwidth}
    \begin{exampleblock}{Esempio (Implementazione)}
\begin{verbatim}
eq(T, X, R), 
        nonvar(T),
        var(X) =>      % 2
    R = [ $eq(X, T) ].
\end{verbatim}
    \end{exampleblock}
  \end{columns}
\end{frame}

%------------------------------------------------
\section{Uso del solver}
%------------------------------------------------

\begin{frame}[fragile]
  \frametitle{Uso del solver}
  \begin{columns}[c]
    \column{.45\textwidth}
    \textbf{Modalità interattiva}
    \begin{exampleblock}{Esempio (Interprete)}
\begin{verbatim}
log> solve 
[eq(X,1), neq(X,1)]
\end{verbatim}
    \end{exampleblock}

    \column{.45\textwidth}
    \textbf{API}
    \begin{exampleblock}{Esempio (API)}
\begin{verbatim}
import solver.

my_predicate =>
    Formula = [ 
        $eq(X,1), 
        $neq(X,1) 
    ],
    solve(Formula, Result),
    println(Result).
\end{verbatim}
    \end{exampleblock}
  \end{columns}
\end{frame}

%------------------------------------------------
\section{Conclusione e sviluppi futuri}
%------------------------------------------------

\begin{frame}
  \frametitle{Conclusioni} Il risultato principale di questo lavoro di
  tesi è stata la progettazione ed implementazione in Picat di un
  constraint solver per formule insiemistiche.
  
  \bigskip
  
  Per farlo, è stato necessario:
  \begin{itemize}
  \item sperimentare e capire Picat;
  \item progettare ed implementare il risolutore di formule stesso.
  \end{itemize}
\end{frame}

%------------------------------------------------

\begin{frame}
  \frametitle{Sviluppi futuri}
  \begin{itemize}
  \item Aggiungere il supporto alla sintassi nativa di \lset{}.
  \item Estendere il supporto all'intero linguaggio \lset{} con funzioni parziali.
  \item Arricchire e perfezionare la batteria di test per aumentare la
    qualità del codice prodotto.
  \end{itemize}
\end{frame}

%------------------------------------------------

\begin{frame}
  \Huge{\centerline{Grazie per l'attenzione}}
\end{frame}

%-------------------------------------------------------------------

\end{document}
