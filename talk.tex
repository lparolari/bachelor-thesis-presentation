\documentclass[11pt]{article}

\usepackage[utf8]{inputenc}  % italian symbols.
\usepackage[T1]{fontenc}     % define T1 charset for out files.
\usepackage[italian]{babel}  % italian latex typo conventions.
\usepackage{csquotes}        % needed by babel.
\usepackage{amsmath}         % math features.
\usepackage{amsthm}          % math theorems.
\usepackage{amssymb}         % math symbols.
\usepackage{hyperref}        % hypertext support.
\usepackage[normalem]{ulem}  % strikeout text.

\title{
	\textbf{Thesis Talk}
	\footnote{\href{https://github.com/lparolari/bachelor-thesis-presentation}{Slides url: https://github.com/lparolari/bachelor-thesis-presentation}}
	\footnote{\href{https://github.com/lparolari/bachelor-thesis}{Thesis url: https://github.com/lparolari/bachelor-thesis}}
	\footnote{\href{https://github.com/lparolari/setlog-picat}{Project url: https://github.com/lparolari/setlog-picat}}
}
\author{Luca Parolari\footnote{\href{mailto:luca.parolari23@gmail.com}{luca.parolari23@gmail.com}}}
\date{26/09/2019}

\input{macros}
\newcommand*{\nextslide}{\textbf{[avanti]}}
\newcommand*{\fingerpointer}{\textbf{[indicare con il dito]}}
%\newcommand*{\textskip}[#1]{\sout{#1}}

\begin{document}

\maketitle

L'obiettivo del lavoro di tesi che vi presento oggi è la progettazione
e l'implementazione di un risolutore di formule per vincoli
insiemistici. \nextslide{}

In questa presentazione voglio dare una breve panoramica del
linguaggio \lset{}, il linguaggio utilizzato per la gestione di
vincoli insiemistici derivato da \clpset{}. Successivamente presenterò
Picat, il linguaggio ospite sul quale è stato implementato il solver,
per poi passare alla descrizione delle scelte effettuate nella
realizzazione del solver e del suo sistema, problemi incontrati ed
implementazione delle regole di riscrittura, componenti fondamentali
del solver tramite i quali è stato possibile risolvere i vincoli. Si
darà poi un accenno all'utilizzo del solver per concludere con la
considerazioni finali ed i lavori futuri. \nextslide{}

\section*{\lset{}}

\lset{} è un linguaggio basato su vincoli per esprimere e risolvere
formule del primo ordine sull'universo degli insiemi finiti. E' stato
inizialmente concepito come un parte del linguaggio logico basato su
vincoli (CLP) \clpset{}. Fondamentalmente, \lset{} fornisce una forma
molto generale di insiemi, letteralmente ``insiemi ibridi finiti
ereditari non tipizzati'' (mi sono esercitato a casa a dirla in
inglese, senza successo, di conseguenza ve la propongo in
italiano). \sout{Gli insiemi finiti ereditari possono essere altri
  inisemi finiti ereditari; mentre gli insiemi ibridi sono insiemi i
  cui elementi possono essere oggetti di tipo non
  insiemistico. L'utilizzo combinato di queste due categorie di
  insiemi da vita ad una forma molto generale di insieme, dove uno o
  più elementi dell'insieme possono essere variabili.} Questi insiemi
sono manipolati dagli usuali operatori su insiemi (come ad esempio
l'eguaglianza tra insiemi, appartenza, unione, eccetera) che sono
forniti sottoforma di vincoli. \sout{Più in particolare, l'uguaglianza
  tra insiemi è implementata con l'unificazione insiemistica, mentre
  il soddisfacimento dei vincoli è verificato da un solver completo
  tramite una procedura di decisione per formule di \lset{} (questo
  solver è appunto l'oggetto oggetto di questo lavoro).} \nextslide{}

Vediamo rapidamente la sintassi di questo linguaggio: possiamo notare
che è denotata da insiemi piuttosto semplici, abbiamo un insieme
finito $\calF$ di costanti e di simboli di funzione non interpretati
come $1, a, f(1,2), ...$, l'insieme dei predicati come appunto $=$,
unione, disgiunzione... e l'insieme delle variabili
$\calV$. \nextslide{}

Il linguaggio consente di combinare gli insiemi precedentemente
mostrati in termini, vincoli o formule. Di particolare importanza sono
le formule, che non sono altro che congiunzioni o disgiunzioni di
termini e vincoli. \nextslide{}

\lset{} è dotato di un solver che implementa una procedura decisionale
per le forule. \'E completa e utilizza riscritture sintattiche per
semplificare le formule e risolverle.

Questa procedura può terminare con esito negativo (nessuna soluzione
trovata) o con esito positivo (una o più soluzioni trovate).
\nextslide{}

\section*{Picat}

Esistono già diverse implementazioni di \lset{}, come ad esempio
setlog scritto in prolog o JSetL per Java. In questo contesto è stato
sperimentato Picat per l'implementazione di \lset{}. Picat è un
linguaggio di programmazione multiparadigna basato sulla logica dei
predicati del primo ordine, per applicazioni general purpose. \'E un
linguaggio piuttosto recente e tengo a sottolineare che questa slide è
già out-to-date, in quanto l'ultima versione non è la 2.6 ma la 2.7,
uscita pochissimi giorni fa, che oltre a varie migliorie aggiunge un
modulo per le reti neurali. \nextslide{}

Picat prende il nome dalle sue principali caratteristiche. Senza
scendere troppo nei dettagli vediamo che in Picat predicati e funzioni
sono definiti tramite pattern-matching, ovvero le regole applicabili
ad una certa chiamata sono seleziona con pattern-matching. Si propone
come intuitivo nel senso che cerca di semplificare le operazioni di
scripting fornendo comandi comunemente utilizzati nei linguaggi
imperativi. Offre supporto nativo alla programmazione con vincoli e
agli attori event-driven. Mette a disposizione un potente meccanismo
di caching per la memorizzazione di risultati parziali. \nextslide{}

Vediamo quindi un esempio di un programma Picat: qui è mostrato come
Picat può essere utilizzato come un normale linguaggio logico con
fatti, predicati, ricorsione... \nextslide{} mentre con un approccio
imperative vediamo come Picat può assumere sembianze diverse con 
funzioni, costrutti di controllo... permettendo al programmatore
l'utilizzo appunto del paradigma imperativo.

Non mi soffermo di troppo sugli esempi perché sono solo a scopo
dimostrativo. \nextslide{}

\medskip

Voglio invece spendere qualche parola sulle differenze tra Picat e
Prolog, il suo rivale. Le differenze tra i due linguaggi non sono
poche e nemmeno trascurabli. questa tabella \fingerpointer{} cerca di
riassumle. Dato che ci sono un paio di esempi nelle slide successive
per selezione delle regole applicabili, backtracking e costrutti
partirei dai predicati dinamici, che a Picat mancano: non è possibile
asserire predicati durante l'esecuzione. Ciò rende Picat meno
flessibile rispetto al Prolog, ma anche meno soggetto a bug. Anche gli
operatori definiti da utente mancano in Picat: questo è un grosso
punto a sfavore perché ad esempio per il linguaggio \lset{} sarebbe
stato molto naturale definire degli operatori per rappresentare i
vincoli. \nextslide{}

Vediamo quindi le differenze nella selezione delle regole applicabili.
In Prolog si utilizza solamente l'unificazione, mentre in Picat
unicamente il pattern-matching. Vediamo infatti che l'esecuzione della
chiamata p(1,Z) comporta divergente tra i due linguaggi: in Prolog ha
successo e unifica Z ad 1, in Picat invece fallisce. Per riprodurre lo
stesso comportamento in Picat il programmatore deve forzare
l'unificazione.  \nextslide{}

Anche sul backtracking trovaimo qualche discrepanza. In prolog tutti i
predicati sono backtrackabili, ed il cotrollo è effettuato con il
cut. In Picat invece il backtracking è esplicito e va indicato nella
testa del predicato con il simbolo punto di domanda.  \nextslide{}

Altra differenza sostanziale tra i due linguaggi sono i comandi e
costrutti di controllo. Picat offre costrutti come l'if then else,
while, foreach e comandi come l'assegnamento che in molti casi
semplificano il codice scritto.  \nextslide{}

\section*{\lset{} in Picat}

Una volta capito che Picat poteva essere un buon linguaggio ospite
tramite il quale implementare il solver è iniziata la fase di
progettazione del sistema.

Come prima cosa è stato necessario definire un linguaggio concreto di
input da dare in paso al solver: la sintassi \lset{} non è
implementabile in Picat senza una fase di parsing, che abbiamo voluto
evitare per problemi di tempo. Le notazioni adottate, riportate sulla
slide. sono semplici espedienti notazionali che ci hanno permesso di
implementare lset in Picat senza grossi sforzi. \nextslide{} E' stato
poi necessario progettare il solver ed il suo sistema: grazie alla
caratteristica di Picat abbiamo potuto separare il sistema in moduli,
ovvero insiemi di funzionalità fortemente accoppiate che interagiscono
con gli altri moduli in modo debolemnte accoppiato. Si è cercato per
quanto possibile di seguire un approccio OOP, integrando nei moduli
delle interfacce offerte all'esterno e mascherando le integrazioni
interne.  \nextslide{}

Il modulo principale di tutto il sistema è ovviamente il solver, che
realizza il ciclo di riscrittura dei vincoli. Di fatto il solver
implementa la procedura \satset{} e restituisce le soluzioni trovate
se esse esisono. Una peculiarità del solver è l'elencare tutte le
soluzioni esistenti, che dalla teoria sappiamo essere finite. Una
volta raggiunto il punto fisso la formula trovata è irriducibile,
semplificata, ed è dimostrabile essere equivalente alla formula
orginale.  \nextslide{}

Il solver possiede un predicato come interfaccia del modulo: il
predicato solve due. Questo implementa il ciclo di riscrittura dei
vincoli, che nel foreach tramite il predicato sat due seleziona le
regole di riscrittura e riscrive i vincoli singolarmente e poi, se il
punto fisso non è stato raggiunto allora la procedura termina,
altrimenti si prosegue riscrivendo il nuovo constraint store.
\nextslide{}

\section*{Implementazione}

Ogni vincolo ha il suo set di regole di riscrittura, ad esempio nella
slide è riportata la regola di riscrittura 2 per il vincolo di
uguaglianza, che ribalta l'uguaglianza se è della forma termine uguale
a variabile. Vediamo che l'implementazione è molto vicina alla
desrcizione matematica: t non appartiene alle variabili si traduce in
nonvar di T e var di X, e la riscrittura cosiste proprio nel vincolo
eq con gli argomenti invertiti.  \nextslide{}

\section*{Uso}

Ma il solver come può essere utilizzato? Abbiamo identificato due
modalità di utilizzo: la prima più a scopi didattici e di debugging è
la modalità interattiva. Viene fornita una semplice CLI tramite la
quale è possibile interagire con il solver. L'altra modalità invece è
l'utilizzo tramite API. Questa modalità consentirebbe l'integrazione
del solver in altri programmi che trattano formule, specialmente
quelle di L SET come un servizio da invocare alla necessità. Il
sistema è ovviamente un po' acerbo e andrebbe perfezionato, ma l'idea
è questa.  \nextslide{}

\section*{Conclusioni e sviluppi futuri}

Quindi, concludento il lavoro realizzato è stato proprio il solver di
\lset{} in Picat. Per farlo è stato necessario sperimentare e capire
un linguaggio logico nuovo ed innovativo come Picat, per poi
progettare ed implementare il solver.  \nextslide{}

Il lavoro non è ovviamente concluso qua: tra le varie migliorie e
sviluppi futuri voglio citare l'aggiunta del supporto alla sintasi
nativa di \lset{} che consentire l'utilizzo vero e proprio del solver,
l'aggiunta del supporto all'intero linguaggio \lset{} attualmente
supportato da setlog e jsetl, ela messa a punto del testing per il
solver che permetterebbe di avere un certo livello di sicurezza sulla
correttezza dell'implementazione. \nextslide{}

Grazie.

\end{document}
