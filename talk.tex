\documentclass[11pt]{article}

\usepackage[utf8]{inputenc}  % italian symbols.
\usepackage[T1]{fontenc}     % define T1 charset for out files.
\usepackage[italian]{babel}  % italian latex typo conventions.
\usepackage{csquotes}        % needed by babel.
\usepackage{amsmath}         % math features.
\usepackage{amsthm}          % math theorems.
\usepackage{amssymb}         % math symbols.
\usepackage{hyperref}        % hypertext support.

\title{
	\textbf{Thesis Talk}
	\footnote{\href{https://github.com/lparolari/bachelor-thesis-presentation}{Slides url: https://github.com/lparolari/bachelor-thesis-presentation}}
	\footnote{\href{https://github.com/lparolari/bachelor-thesis}{Thesis url: https://github.com/lparolari/bachelor-thesis}}
	\footnote{\href{https://github.com/lparolari/setlog-picat}{Project url: https://github.com/lparolari/setlog-picat}}
}
\author{Luca Parolari\footnote{\href{mailto:luca.parolari23@gmail.com}{luca.parolari23@gmail.com}}}
\date{26/09/2019}

\input{macros}

\begin{document}

\maketitle

L'obiettivo del lavoro di tesi che vi presento oggi è la progettazione
e l'implementazione di un risolutore di formule per vincoli
insiemistici.

In questa presentazione voglio dare una breve panoramica del
linguaggio \lset{}, il linguaggio utilizzato per la gestione di
vincoli insiemistici derivato da \clpset{}. Successivamente presenterò
Picat, il linguaggio ospite sul quale è stato implementato il solver,
per poi passare alla descrizione delle scelte effettuate nella
realizzazione del solver e del suo sistema, problemi incontrati ed
implementazione delle regole di riscrittura, componenti fondamentali
del solver tramite i quali è stato possibile risolvere i vincoli. Si
darà poi un accenno all'utilizzo del solver per concludere con la
considerazioni finali ed i lavori futuri.

TODO

\end{document}